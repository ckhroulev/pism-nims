\documentclass[hide notes,intlimits]{beamer}

\mode<presentation>
{
  \usetheme[footline]{PISMshade}
  \setbeamercovered{transparent}
}

% load packages
\usepackage{multimedia}
\usepackage[english]{babel}
\usepackage[utf8]{inputenc}
\usepackage[T1]{fontenc}
\usepackage{lmodern}
\usepackage{pdfpages}
\usepackage[multidot]{grffile}
\usepackage{listings}

\definecolor{mygreen}{rgb}{0,0.6,0}
\definecolor{mygray}{rgb}{0.5,0.5,0.5}
\definecolor{mymauve}{rgb}{0.58,0,0.82}

\lstset{ 
  backgroundcolor=\color{white},   % choose the background color; you must add \usepackage{color} or \usepackage{xcolor}; should come as last argument
  basicstyle=\tt\tiny,        % the size of the fonts that are used for the code
  breakatwhitespace=false,         % sets if automatic breaks should only happen at whitespace
  breaklines=true,                 % sets automatic line breaking
  captionpos=b,                    % sets the caption-position to bottom
  commentstyle=\color{mygreen},    % comment style
  deletekeywords={...},            % if you want to delete keywords from the given language
  escapeinside={\%*}{*)},          % if you want to add LaTeX within your code
  extendedchars=true,              % lets you use non-ASCII characters; for 8-bits encodings only, does not work with UTF-8
  frame=single,	                   % adds a frame around the code
  keepspaces=true,                 % keeps spaces in text, useful for keeping indentation of code (possibly needs columns=flexible)
  keywordstyle=\color{blue},       % keyword style
  language=C++,                    % the language of the code
  morekeywords={*,...},            % if you want to add more keywords to the set
  numbers=left,                    % where to put the line-numbers; possible values are (none, left, right)
  numbersep=5pt,                   % how far the line-numbers are from the code
  numberstyle=\tiny\color{mygray}, % the style that is used for the line-numbers
  rulecolor=\color{black},         % if not set, the frame-color may be changed on line-breaks within not-black text (e.g. comments (green here))
  showspaces=false,                % show spaces everywhere adding particular underscores; it overrides 'showstringspaces'
  showstringspaces=false,          % underline spaces within strings only
  showtabs=false,                  % show tabs within strings adding particular underscores
  stepnumber=2,                    % the step between two line-numbers. If it's 1, each line will be numbered
  stringstyle=\color{mymauve},     % string literal style
  tabsize=2,	                   % sets default tabsize to 2 spaces
  title=\lstname                   % show the filename of files included with \lstinputlisting; also try caption instead of title
}

\usepackage{tikz}
\usetikzlibrary{shapes,arrows}
\usetikzlibrary{shadows}
\usetikzlibrary{mindmap}

\definecolor{dark red}{HTML}{E41A1C}
\definecolor{dark green}{HTML}{4DAF4A}
\definecolor{dark violet}{HTML}{984EA3}
\definecolor{dark blue}{HTML}{084594}
\definecolor{dark orange}{HTML}{FF7F00}
\definecolor{light blue}{HTML}{377EB8}
\definecolor{light red}{HTML}{FB9A99}
\definecolor{light violet}{HTML}{CAB2D6}

\setbeamercolor{boxed}{fg=black,bg=light blue!25}
\graphicspath{{figures/}}

\setbeamertemplate{background canvas}
{
  \tikz{\node[inner sep=0pt,opacity=1.0]
    {\includegraphics[width=\paperwidth]{pism_beamer_shade_bg}};}
}

\newcommand{\diff}[2]{\frac{\partial #1}{\partial #2}}

% title page
\title{Customizing and extending PISM: the basics}

\author{C.~Khrulev}
\institute{University of Alaska Fairbanks}
\date{}

\subject{The Parallel Ice Sheet Model}

\begin{document}

% insert titlepage
\begin{frame}
  \titlepage
\end{frame}

\setbeamertemplate{background canvas}{}

\section{Sub-models}
\label{sec:sub-models}

\begin{frame}
  \frametitle{PISM consists of...}

  % PuBu
  \definecolor{levelo}{RGB}{5,112,176}
  \definecolor{leveli}{RGB}{116,169,207}
  \definecolor{levelii}{RGB}{189,201,225}
  \definecolor{leveliii}{RGB}{241,238,246}

  \begin{center}
    \scalebox{0.55}{
      \begin{tikzpicture}[mindmap, concept color=levelo, font=\sf, text=black]

        \tikzstyle{level 1 concept}+=[font=\sf, sibling angle=40]

        \node[concept] {PISM core managing initialization, time stepping, passing data between sub-models, saving requested diagnostic quantities, etc}
        [clockwise from=0]
        child[concept color=leveli]{
          node[concept]{Geometry evolution}
          [clockwise from=60]
          child[concept color=levelii]{ node[concept]{Mass transport} }
          child[concept color=levelii]{ node[concept]{Calving} }
          child[concept color=levelii]{ node[concept]{Iceberg elimination} }
        }
        child[concept color=leveli]{
          node[concept]{Energy balance}
          [clockwise from=-40]
          child[concept color=levelii]{ node[concept]{Bedrock thermal layer}}
        }
        child[concept color=leveli]{ node[concept]{Stress balance} }
        child[concept color=leveli]{ node[concept]{Bed deformation} }
        child[concept color=leveli]{ node[concept]{Age} }
        child[concept color=leveli]{ node[concept]{Subglacial hydrology} }
        child[concept color=leveli]{ node[concept]{Sea level forcing} }
        child[concept color=leveli]{ node[concept]{Sub-shelf boundary conditions} }
        child[concept color=leveli]{
          node[concept]{Top surface boundary conditions}
          [clockwise from=40]
          child[concept color=levelii]{ node[concept]{Atmosphere model} }
        }
        ;
      \end{tikzpicture}
    }                           %scalebox
  \end{center}
\end{frame}

\begin{frame}
  \frametitle{Components}

  Most sub-models in PISM are derived from the ``Component'' class.

  They need to be able to
  \begin{itemize}
  \item initialize
  \item compute maximum time step
  \item update the state (``step'')
  \item ``define'' the model state (NetCDF variables)
  \item write the model state to a file
  \item<2> provide scalar diagnostic quantities
  \item<2> provide spatially-variable diagnostic quantities
  \end{itemize}
  
\end{frame}

\begin{frame}
  \frametitle{Initialization}

  Two kinds of initializaion:
  \begin{itemize}
  \item re-starting from a saved state
  \item starting from a (possible) incomplete set of input data
    (``bootstrapping'')
  \end{itemize}
\end{frame}

\begin{frame}[fragile,fragile]
  \frametitle{Initialization: an example}
\begin{lstlisting}
void ConstantYieldStress::init_impl() {
  m_log->message(2, "* Initializing the constant basal yield stress model...\n");

  InputOptions opts = process_input_options(m_grid->com, m_config);
  const double tauc = m_config->get_double("basal_yield_stress.constant.value");

  switch (opts.type) {
  case INIT_RESTART:
    m_basal_yield_stress.read(opts.filename, opts.record);
    break;
  case INIT_BOOTSTRAP:
    m_basal_yield_stress.regrid(opts.filename, OPTIONAL, tauc);
    break;
  case INIT_OTHER:
  default:
    // Set the constant value.
    m_basal_yield_stress.set(tauc);
  }

  regrid("ConstantYieldStress", m_basal_yield_stress);
}
\end{lstlisting}
\end{frame}

\begin{frame}
  \frametitle{PISM's computational grid}
  
\end{frame}

\begin{frame}[fragile]
  \frametitle{Working with 2D and 3D fields}

  Without ghosts:
\begin{lstlisting}
IceModelVec2S temperature(grid, "air_temp", WITHOUT_GHOSTS);

IceModelVec::AccessList access{&temperature};

for (Points p(grid); p; p.next()) {
  const int i = p.i(), j = p.j();

  temperature(i, j) = i + j;
}
\end{lstlisting}

  With ghosts:
\begin{lstlisting}
IceModelVec2S temperature(grid, "air_temp", WITH_GHOSTS, 1);

IceModelVec::AccessList access{&temperature};

for (PointsWithGhosts p(grid); p; p.next()) {
  const int i = p.i(), j = p.j();

  temperature(i, j) = i + j;
}
\end{lstlisting}
  
\end{frame}

\end{document}